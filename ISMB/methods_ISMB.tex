%!TEX root = main_ISMB.tex
\section{Methods}
\label{sec:methods}



We introduce a probabilistic model for the design of RNA sequences with a specific \GCContent and folding into a predefined secondary structure.
For the sake of simplicity, we choose to base this proof-of-concept implementation on a simplified free-energy function $\ES(\cdot)$, which only considers the contributions of 
stacked canonical base-pairs. We show how a modification of the dynamic programming scheme used in \RNAmutants allows for the sampling of good and diverse design candidates, in linear time and space complexities.


%To that purpose, a Boltzmann weighted distribution is used, based on an energy function $\PE{\cdot}$ which includes contributions for both the free-energy and its putative isostericity towards a multiple sequence alignment. In this model, the probability that the nucleotide at a given position needs to be mutated (i.e. corresponds to a sequencing error) can be computed using a variant of the \emph{Inside-Outside algorithm}~\citep{Lari1990}.
%
%\subsection{Probabilistic model}
%Let $\Omega$ be an gap-free RNA alignment sequence, $S$ its associated secondary structure, 
%then any sequence $s$ is assigned a probability proportional to its Boltzmann factor
%\begin{align*}
%  \mathcal{B}(s) &= e^\frac{-\PE{s}}{RT}, &&\text{with}&\PE{s}&:=\alpha\cdot\ES(s,S)+(1-\alpha)\cdot\EI(s,S,\Omega),
%\end{align*}
%where $R$ is the Boltzmann constant, $T$ the temperature in Kelvin, $\ES(s)$ and $\EI(s,S,\Omega)$ 
%are the free-energy and isostericity contributions respectively (further described below), and $\alpha\in[0,1]$ is an arbitrary parameter that sets the relative weight for both contributions.
%
%\subsubsection{Energy contribution}
\subsection{Definitions}

A targeted secondary structure $\Target$ of length $n$ is given as a non-crossing arc-annotated sequence,  where 
$\Target_i$ stands for the base-pairing position of position $i$ in $\Target$ if any (and, reciprocally, $\Target_{\Target_i}=i$), or $-1$ otherwise. 
In addition, let us denote by $\gc(s)$ the number of occurrences of \Gb and \Cb in an RNA sequence $s$.

\subsubsection{Simplified energy model}
We use a simplified free-energy model which only includes additive contributions from stacking base-pairs. Using individual values from the Turner 2004 model (retrieved from the NNDB~\citep{Turner2010}). Given a candidate sequence $s$ for a secondary structure $\Struct$, the free-energy of any sequence $s$ of length $|\Struct|$  is given by
\begin{align*}
  \ES(s,\Struct) = \sum_{\substack{(i,j)\to (i',j')\in \Struct\\ \text{stacking pairs}}}\ES^{\beta}_{s_is_j\to s_{i'}s_{j'}} 
\end{align*}
where $\ES^{\beta}_{ab\to a'b'}$ is set to $0$ if $ab=\varnothing$ (no base-pair to stack onto), the tabulated free-energy of stacking pairs $(ab)/(a'b')$ in the Turner model if available, or $\beta\in[0,\infty]$ for non-Watson-Crick/Wobble pairs (i.e. not in $\{\Gb\Ub,\Ub\Gb,\Cb\Gb,\Gb\Cb, \Ab\Ub\text{ or }\Ub\Ab\}$). This latter parameter allows one to choose whether to simply penalize invalid base pairs ($\beta>0$), or forbid them altogether ($\beta = +\infty$). 
Position-specific sequence constraints can also be enforced at this level (details omitted for the sake of clarity) by assigning to $\ES$ a $+\infty$ penalty (leading to a null probability) in the presence of a base incompatible with a user-specified constraint mask.


\subsubsection{\GCWeighted Boltzmann ensemble and distribution}

In order to counterbalance the documented tendency of sampling methods to generate \Gb\Cb-rich sequences~\citep{Levin:2012kx}, we introduce a parameter $x\in\mathbb{R}^+$, whose value will influence the \GCContent of generated sequences. For any secondary structure $\Struct$, the \GCWeighted-Boltzmann factor of a sequence $s$ is $\B^{[x]}_{\Struct}(s)$  such that
\begin{equation}
\B_{\Struct}^{[x]}(s) = e^{\frac{-\ES(s,\Struct)}{RT}}\cdot x^{\gc(s)}
\label{def:genBoltz}
\end{equation}
where $R$ is the Boltzmann constant and $T$ the temperature in Kelvin.

Summing the \GCWeighted-Boltzmann factor over all possible sequences of a given length $|\Struct|$, one obtains the \GCWeighted partition function $\mathcal{Z}_{\Struct}^{[x]}$, from which one defines the \GCWeighted Boltzmann probability $\Prob_{\Struct}^{[x]}(s)$ of each sequence $s$, respectively such that 
\begin{align}\mathcal{Z}_{\Struct}^{[x]} &= \sum_{\substack{|s|=n}}\B_{\Struct}^{[x]}(s)& \text{and}&&
\Prob_{\Struct}^{[x]}(s) &= \frac{\B_{\Struct}^{[x]}(s)}{\mathcal{Z}_{\Struct}^{[x]}}.\label{def:distribution}\end{align}

\subsection{Linear-time stochastic sampling algorithm for the \GCWeighted-Boltzmann ensemble}

Let us now describe a linear-time algorithm to sample sequences at random in the \GCWeighted Boltzmann distribution. This algorithm follows the general principles of the recursive approach to random generation~\citep{Wilf1977}, pioneered in the context of RNA by the \SFold algorithm~\citep{Ding2003}. The algorithm starts by precomputing the partition function restricted to each substructure occurring in the target structure, and then performs a series of recursive stochastic backtracks, using precomputed values to decide on the probability of each alternative. 

\highlight{
\subsubsection{Precomputing the \GCWeighted partition function}\label{sec:pf}

Firstly, a dynamic programming algorithm computes $\Z{\N,\Struct}{a,b}$ the \GCWeighted partition function (the dependency in $x$ is omitted here for the sake of clarity) for a structure $\Struct$, assuming its (previously chosen) flanking nucleotides are $a$ and $b$ respectively, either forming a closing base-pair ($\N=\BoolTrue$) or not ($\N=\BoolFalse$).
Remark that the empty structure only supports the empty sequence, having energy $0$, so one has
\begin{equation}
	\Z{\BoolTrue,\varepsilon}{a,b}=\Z{\BoolFalse,\varepsilon}{a,b}= e^{-0/RT} = 1.
	\label{eq:Z_in}
\end{equation}


The general recursion scheme consists in three different terms, depending on the first position in $\Struct$:\\
{\bf Case 1.} First position is unpaired ($\Struct = \ub \Struct'$):
\begin{equation}
	\Z{\BoolTrue,\ub \Struct'}{a,b} = \Z{\BoolFalse,\ub \Struct'}{a,b}  :=
      \sum_{a'\in \B}  
      x^{\gc(a')}
      \cdot\Z{\BoolFalse,\Struct'}{a',b}; 
\label{eq:Z_unpaired}
\end{equation}
{\bf Case 2.} First position is paired with last position ($\Struct = \op \Struct' \cp$), stacking onto a pre-existing exterior pair ($\N=\BoolTrue$):
\begin{equation}
	\Z{\BoolTrue,\op \Struct' \cp}{a,b} :=
      \sum_{a',b'\in \B^2}
			 x^{\gc(a'.b')}
			 \cdot e^{\frac{-\ES^{\beta}_{ab \to a'b'}}{RT}}
			 \cdot \Z{\BoolTrue,\Struct'}{a',b'};
\label{eq:Z_stack}
\end{equation}
{\bf Case 3.} First position is involved in a base-pair ($\Struct = \op \Struct' \cp \Struct''$), which is not stacking onto an exterior base-pair ($\N=\BoolFalse$ or $\Struct''\neq \varepsilon$):
\begin{equation}
	\Z{\N,\op \Struct' \cp \Struct''}{a,b} :=
			 \displaystyle
      \sum_{a',b'\in \B^2}
      x^{\gc(a'.b')}
			\cdot e^{\frac{-\ES^{\beta}_{\varnothing\to a'b'}}{RT}}
      \cdot\Z{\BoolTrue,\Struct'}{a',b'}
      \cdot\Z{\BoolFalse,\Struct''}{b',b}.
\label{eq:Z_rec}
\end{equation}

Remark that the number of combinations of $a$, $b$ and $\N$ remains bounded by a constant, thus the complexity of computing $\Z{\N,\Struct}{a,b}$ mainly depends on the values taken by $\Struct$ upon subsequent recursive calls. Such values are entirely determined by $\Struct$ at any given step of the recursion, and their dependency can be summarized in a tree having $\Theta(|\Struct|)$. Therefore, the computation of $\Z{\N,\Target}{a,b}$ requires $\Theta(n)$ time and space using dynamic-programming.}

\begin{figure*}
\resizebox{\linewidth}{!}{
\begin{tikzpicture}

  \newcommand{\BSep}{9pt}
  \newcommand{\HSep}{280pt}
  \newcommand{\VSep}{85pt}
  \tikzstyle{base}=[circle,draw,thick,inner sep=0,minimum width=15pt,fill=white]
  \tikzstyle{basesmall}=[circle,draw,thick,inner sep=0,minimum width=10pt,fill=white]
  \tikzstyle{basephantom}=[base,dashed]
  \tikzstyle{linez}=[draw,snake=coil, segment aspect=.2,%
line after snake=0pt, 
        segment length=10pt,thick]
  \tikzstyle{lined}=[linez,draw,snake=none,thick]
  \tikzstyle{line}=[linez,draw,snake=none,thick]
  \tikzstyle{bp}=[in=90,out=90,draw,line width=1.5pt,blue,looseness=1]
  \tikzstyle{block}=[trapezium,trapezium angle=33, fill=green!20, draw=green!40!black,line width=1.5pt, inner sep=0]
  \tikzstyle{lbl}=[inner sep=0]
  \tikzstyle{arr}=[line width=1.3pt,decorate,decoration={snake,amplitude=1mm, segment length=20pt,pre length=1mm, post length=2mm},->]

  \node[base] (a-1) at (0,0) {$s_i$};
  \node[base] (a-2) at (3,0) {$s_j$};
  \node[left=\BSep of a-1, basephantom] (a-0) {$a$};
  \node[right=\BSep of a-2, basephantom] (a-3) {$b$};
  \node[right=0 of a-3] (x) {};


  \path[linez] (a-1) --  (a-2);
  \path[lined] (a-0) -- (a-1);
  \path[lined] (a-2) -- (a-3);
  \node[lbl,below=3pt of a-1] (c1) {$i$};
  \node[lbl,below=3pt of a-2] (c2) {$j$};



  \begin{pgfonlayer}{background}
  \node[rectangle,inner sep=2pt,draw,fit=(a-1.west)(a-2.east) (c1) (c2)] (r1) {};
  \path[block]   (r1.south west) to (r1.south east) to (r1.north east) to[out=110,in=70] (r1.north west) to (r1.south west) ;
  \end{pgfonlayer}{background}


\begin{scope}[xshift=\HSep,yshift=\VSep]
  \node[base] (a-1) at (0,0) {\color{red}$a'$};
  \node[basesmall,right=\BSep of a-1] (a-1b)  {};
  \node[basesmall] (a-2) at (5,0) {};
  \node[left=\BSep of a-1, basephantom] (a-0) {$a$}; 
  \node[right=\BSep of a-2, basephantom] (a-3) {$b$};
  \path[linez] (a-1b) --  (a-2);
  \path[line] (a-1) -- (a-1b);
  \path[lined] (a-0) -- (a-1);
  \path[lined] (a-2) -- (a-3);
  \node[below=3pt of a-1b, inner sep=0] (c1) {$i+1$};
  \node[below=3pt of a-2, inner sep=0] (c2) {$j$};

  \node[right=3em of a-3,anchor=west] {\relsize{+2}[Position $i$ unpaired]};
  \node[xshift=-2pt] at (a-0.west) (y1) {};

\begin{pgfonlayer}{background}
  \node[rectangle,inner sep=2pt,draw,fit=(a-1b.west)(a-2.east) (c1) (c2)] (r1) {};
  \path[block]   (r1.south west) to (r1.south east) to (r1.north east) to[out=110,in=70,looseness=.6] (r1.north west) to (r1.south west) ;

  \end{pgfonlayer}{background}

\end{scope}

\begin{scope}[xshift=\HSep,yshift=-\VSep]
  \node[base] (a-1) at (0,0) {\color{red}$a'$};
  \node[base] (a-p) at (3,0) {\color{red}$b'$};

  \node[basesmall,right=\BSep of a-1] (a-1b) {};
  \node[basesmall,left=\BSep of a-p] (a-pb)  {};
  \node[basesmall,right=\BSep of a-p] (a-pa)  {};

  \node[basesmall] (a-2) at (5,0) {};
  \node[left=\BSep of a-1, basephantom] (a-0) {$a$};
  \node[right=\BSep of a-2, basephantom] (a-3) {$b$};

  \node[right=3em of a-3,anchor=west] {\relsize{+2}[Position $i$ paired anywhere else]};

  \path[linez] (a-1b) -- (a-pb);
  \path[linez] (a-pa) --(a-2);

  \path[line] (a-1) -- (a-1b);
  \path[line] (a-pb) -- (a-p);
  \path[line] (a-pa) -- (a-p);

  \path[lined] (a-0) -- (a-1);
  \path[lined] (a-2) -- (a-3);
  \node[lbl,below=3pt of a-1b] (c1) {$i+1$};
  \node[lbl,below=3pt of a-2] (c5){$j$};
  \node[lbl,below=3pt of a-p] (c3) {$k$};
  \node[lbl,below=3pt of a-pb] (c2){$k-1$};
  \node[lbl,below=3pt of a-pa] (c4){$k+1$};

  \draw[bp]  (a-1) to (a-p);
  \node[xshift=-2pt] at (a-0.west) (y2) {};

\begin{pgfonlayer}{background}

  \node[rectangle,inner sep=2pt,draw,fit=(a-1b.west)(a-pb.east) (c1) (c2)] (r1) {};
  \path[block]   (r1.south west) to (r1.south east) to (r1.north east) to[out=110,in=70,looseness=0.8] (r1.north west) to (r1.south west) ;

  \node[rectangle,inner sep=2pt,draw,fit=(a-pa.west)(a-2.east) (c4) (c5)] (r2) {};
  \path[block]   (r2.south west) to (r2.south east) to (r2.north east) to[out=110,in=70,looseness=0.8] (r2.north west) to (r2.south west) ;

  \end{pgfonlayer}{background}

\end{scope}

 
\begin{scope}[xshift=\HSep,yshift=00pt]
  \node[base] (a-1) at (0,0) {\color{red}$a'$};
  \node[base] (a-2) at (5,0) {\color{red}$b'$};

  \node[basesmall,right=\BSep of a-1] (a-1b) {};
  \node[basesmall,left=\BSep of a-2] (a-2b)  {};

  \node[left=\BSep of a-1, basephantom] (a-0) {$a$};
  \node[right=\BSep of a-2, basephantom] (a-3) {$b$};

  \path[linez] (a-1b) --  (a-2b);

  \path[line] (a-1) -- (a-1b);
  \path[line] (a-2) -- (a-2b);
  \path[lined] (a-0) -- (a-1);
  \path[lined] (a-2) -- (a-3);
  \node[lbl,below=3pt of a-1b] (c1) {$i+1$};
  \node[lbl,below=3pt of a-2b] (c2){$j-1$};

  \draw[bp,looseness=.7]  (a-1) to (a-2);
  \draw[bp,looseness=.7]  (a-0) to (a-3);
  \node[right=3em of a-3,anchor=west] {\relsize{+2}[Paired ends + Stacking pairs]};
  \node[xshift=-2pt] at (a-0.west) (y3) {};

\begin{pgfonlayer}{background}
  \node[rectangle,inner sep=2pt,draw,fit=(a-1b.west)(a-2b.east) (c1) (c2)] (r1) {};
  \path[block]   (r1.south west) to (r1.south east) to (r1.north east) to[out=110,in=70,looseness=.6] (r1.north west) to (r1.south west) ;
  \end{pgfonlayer}{background}

\end{scope}

  \tikzstyle{prob}=[fill=white,line width=1pt, opacity=.9]
  \path (x) edge[arr] %node[midway,prob]{$\frac{\mathcal{Z}\substack{(i+1,j)\\ [a',b]}}{\mathcal{Z}\substack{(i,j)\\ [a,b]}}$} 
(y1);
  \path (x) edge[arr] %node[midway,prob]{ $\frac{\mathcal{Z}\substack{(i+1,j)\\ [a',b]}}{\mathcal{Z}\substack{(i,j)\\ [a,b]}}$} 
(y2);
  \path (x) edge[arr] %node[midway,prob]{$\frac{\mathcal{Z}\substack{(i+1,j)\\ [a',b]}}{\mathcal{Z}\substack{(i,j)\\ [a,b]}}$} 
(y3);
  
\end{tikzpicture}}
\caption{Stochastic backtrack procedure for a given substructure $\Struct$: Either the first position is left unpaired (top), a base-pair is formed between the two extremities, stacking onto an exterior base-pair (middle), or paired without creating a stacking, defining two regions on which subsequent recursive calls are needed (bottom). For the empty structure (omitted here), the empty sequence is returned. Positions indicated in red are assigned at the current stage of the backtrack.\label{fig:stochastic}}
\end{figure*}

\begin{algorithm}[t]
\DontPrintSemicolon
	\SetAlgoLined
\SetKwFunction{Backtrack}{SB$_x$}
\SetKwFunction{Random}{Random}
\newcommand{\rand}{{r}}
	$\rand \leftarrow $\Random$\left(\Z{\N,\Struct}{a,b}\right)$\tcp*[r]{Random real in $[0,\Z{\N,\Struct}{a,b}[$}
 \Switch{}{
   \lCase(\tcp*[f]{Empty structure}){$\Struct=\varepsilon$}{\Return{$\varepsilon$}}
   \Case(\tcp*[f]{First position is unpaired}){$\Struct=\ub\, \Struct'$}{		
   \For{$a'\in\B$}{
			$\rand \leftarrow \rand - x^{\gc(a')}\cdot \Z{\BoolFalse,\Struct'}{a',b}$\;
			\lIf{$\rand<0$}\Return{$a'.\Backtrack(a',b,\BoolFalse,\Struct')$}\;
  		}
   }
   \Case(\tcp*[f]{Extremities are involved in stacking base pair}){$\Struct=\text{\rm\op}\, \Struct' \,\text{\rm\cp}$  {\rm \bf and} $\N=\BoolTrue$}
   {		
			\For{$(a',b')\in\B\times\B$}
    {
				$\rand \leftarrow \rand -
			 	x^{\gc(a'.b')}
			 	\cdot e^{{-\ES^{\beta}_{ab \to a'b'}}/{RT}}
			 	\cdot \Z{\BoolTrue,\Struct'}{a',b'}	$\;
				\lIf{$\rand<0$}{
					\Return{$a'.\Backtrack(a',b',\BoolTrue,\Struct'). b'$}\;		
				}
			}
		}
   \Other(\tcp*[f]{First position is paired without a stacking pair})
   {
				\tcp{$\Struct=\text{\rm\op}\, \Struct' \,\text{\rm\cp}\,\Struct''$}
				\For{$(a',b')\in\B\times\B$}{
					$\rand \leftarrow \rand -	
       x^{\gc(a'.b')}
					\cdot e^{\frac{-\ES^{\beta}_{\varnothing\to a'b'}}{RT}}
  	    	\cdot\Z{\BoolFalse,\Struct'}{a',b'}
	    	  \cdot\Z{\BoolTrue,\Struct''}{b',b}$\;	
 					\lIf{\rand $<0$}{
						\Return{$a'
						.\Backtrack\left(a',b',\BoolTrue,\Struct'\right)
						.b'
						.\Backtrack\left(b',b,\BoolFalse,\Struct''\right)
						$}\;	
					}
				}
   }
 }
\caption{\protect\Backtrack$\left(a,b,\N,\Struct\right)$\label{alg:back}}
\end{algorithm}
\subsubsection{Stochastic backtrack}
Once the \GCWeighted partition functions have been computed and memorized, a stochastic backtrack starts from the target structure $\Target$ with any exterior bases $[a,b]$ and no nesting base-pair, corresponding to a call \Backtrack$\left(\varnothing,\varnothing,\BoolFalse,\Target\right)$ to Algorithm~\ref{alg:back}.  At each step, a suitable assignment for one or several positions is chosen, using probabilities derived from the precomputation, as illustrated by Figure~\ref{fig:stochastic}. One or several recursive calls over the appropriate substructures are then performed. On each recursive call, the algorithm assigns at least one nucleotide to a -- previously unassigned -- position. Moreover, the number of executions of each loops is bounded by a constant. Consequently, the complexity of Algorithm~\ref{alg:back} is in $\Theta(n)$ time and space. 



\begin{figure}
\resizebox{\linewidth}{!}{
\tikzstyle{decision} = [diamond, draw, inner color=blue!10, outer color=blue!20!white!80!black, inner sep = 1pt,drop shadow]
\tikzstyle{decision answer}=[midway, color=black]
\tikzstyle{block} = [rectangle, draw,
outer color=gray!20,
inner color=white,
text width=4.3cm, minimum height=1cm,drop shadow,
thick]
\tikzstyle{edgeblock} = [block, rectangle split,rectangle split parts=2, rounded corners=10pt,
outer color=white,
inner color=white,minimum height=2cm]
\tikzstyle{line} = [draw, very thick, color=black!50, -latex']
\tikzstyle{cloud} = [draw, ellipse,fill=red!20]


\begin{tikzpicture}[node distance = 2cm, auto, font=\sffamily\small,text centered,]
    % Place nodes
  \newcommand{\res}{Samples}
  \newcommand{\VDist}{.6cm}
    \node [edgeblock] (init) {{\normalsize Start}\nodepart{second}{} {$x:=1$\quad $ \res:=\varnothing$}};
    \node [block,below=\VDist of init] (gen) {Draw sequences in\\ $x$-weighted distribution};
    \node [block, below=\VDist of gen] (filter) {Filter on GC$\%$\\ Add suitable seqs to $\res$};
    \node [block, right=0.6cm of filter, node distance=2.5cm] (update) {Estimate average GC\% for $x$\\ Update $x$};
    \node [decision, below=\VDist of filter] (decide) {$|\res| \ge k$?};
    \node [edgeblock, below=\VDist of decide] (stop) {{Stop}\nodepart{second}{}{Return $\res$}};
    % Draw edges
    \path [line] (init) -- (gen);
    \path [line] (gen) -- (filter);
    \path [line] (filter) -- (decide);
    \path [line] (update) |- (gen);
    \path [line] (decide) -| node[decision answer,near start,below] {no} (update);
    \path [line] (decide) -- node[decision answer,right] {yes} (stop);
\end{tikzpicture}
}
\caption{General workflow of our adaptive sampling algorithm~\citep{Waldispuhl2011}.}
\label{fig:workflow}
\end{figure}

\highlight{
\subsubsection{Self-adaptive sampling strategy}

Let us remind that our goal is to produce a set of sequences whose \GCContent matches a prescribed value $gc$.
An absolute tolerance $\kappa$ may be allowed, so that the \GCContent of any valid sequence must fall in  
$[gc-\kappa,gc+\kappa]$. Since sequences of arbitrary \GCContent may be generated by Algorithm~\ref{alg:back}, we use a rejection-based approach~\citep{Bodini2010}, previously adapted by the authors in a similar context~\citep{Waldispuhl2011}. This gives an algorithm which generates $k$ valid sequences in expected time $\Theta(k\cdot n\sqrt{n})$ when $\kappa=0$ (or $\Theta(k\cdot n)$ when $\kappa$ is a positive constant) and memory in $\Theta(k\cdot n)$.
A complete analysis of the rejection process can be found in an earlier contribution~\citep{Waldispuhl2011}, but let us briefly outline the approach, and the main arguments used to establish its complexity.


As summarized by Figure~\ref{fig:workflow}, our adaptive sampling approach simply generates sets of sequences by repeatedly running the stochastic backtrack algorithm. The average \GCContent induced by the current value of the $x$ parameter, can then be adequately estimated from the sample, or computed exactly using recent algorithmic advances~\citep{Ponty2011}. The set of sequences is filtered to only retain valid sequences. The value of the parameter $x$ is then adapted to match the average \GCContent (induced by the value of $x$) with the targeted one.
It can be shown that the expected \GCContent is a continuous and strictly increasing monotonic function of $x$, whose limits are $0$ when $x=0$ and $n$ when $x\to +\infty$. Consequently, for any targeted \GCContent $gc\in[0\%,100\%]$, there exists a unique value $x_{gc}$ such that generated sequences feature, on the average, the right \GCContent.
In practice, a simple binary search~\citep{Waldispuhl2011} is used in our implementation, and typically converges after very few iterations. An optimal value for $x$ can also be derived analytically using interpolation after $\Theta(n)$ evaluations of $\Z{i,j}{a,b}$ for different candidate values of $x$, as previously noted~\citep{Waldispuhl2011} and could be implemented using the Fast-Fourier Transform~\citep{Senter2012}.

\subsubsection{Overall complexity}

It was previously established~\citep{Waldispuhl2011} that, for each value of $x$, there exists constants $\mu_x$ and $\sigma_x$ such that the distribution of \GCContent asymptotically converges towards a normal law having expectation in $\mu_x\cdot n\cdot(1+o(1))$ and standard deviation in $\sigma_x\cdot\sqrt{n}\cdot(1+o(1))$.
Furthermore, the distribution of \GCContent is highly concentrated, as asserted by its limited standard deviation, therefore the expected number of attempts required to generate a valid sequence when $\kappa=0$ (resp. $\kappa\in\Omega(1/\sqrt n)$) grows like $\Theta(\sqrt{n})$ (resp. $\Theta(1)$, i.e. a constant), leading to the announced complexities. Formally, since a suitable weight $x$ must be recomputed for each targeted structure and \GCContent, then the number $M$ of iterations required for the converge can be accounted for explicitly, leading to time complexities in $\Theta((M+\sqrt{n})\cdot k\cdot n)$ (if $\kappa=0$, i.e. without any tolerance) and $\Theta(M\cdot k\cdot n)$ (if $\kappa>0$).
}


\subsection{Postprocessing unpaired regions: A local/global (glocal) hybrid approach}
\label{subsec:glocal_method}
Due to our simplified energy model, unpaired regions are not subject to design constraints other than the \GCContent, leading to modest probabilities for refolded design candidates to match the targeted structure. To improve these performances and test the complementarity  of our global sampling approach with previous contributions based on local search, we used the \RNAinverse software to redesign unpaired regions. We specified a constraint mask to prevent stacking base-pairs from being modified and, whenever necessary, reestablished their content {\em a posteriori}, as \RNAinverse has been witnessed to take some liberties with constraints masks. As shown in Table~\ref{table:impact_on_gc} (Supplementary material), this postprocessing does not drastically alter the \GCContent, so the glocal approach reasonably addresses the constrained \GCContent design problem.


