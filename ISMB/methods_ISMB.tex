%!TEX root = main_ISMB.tex
\section{Methods}
\label{sec:methods}



We introduce a probabilistic model for the design of RNA sequences with a specific \GCContent and folding into a predefined secondary structure.
For the sake of simplicity, we choose to base this proof-of-concept implementation on a simplified free-energy function $\ES(\cdot)$, which only considers the contributions of 
stacked canonical base-pairs. We show how a modification of the dynamic programming scheme used in the \RNAmutants allows for the sampling of good and diverse design candidates, in linear time and space complexities.


%To that purpose, a Boltzmann weighted distribution is used, based on a pseudo-energy function $\PE{\cdot}$ which includes contributions for both the free-energy and its putative isostericity towards a multiple sequence alignment. In this model, the probability that the nucleotide at a given position needs to be mutated (i.e. corresponds to a sequencing error) can be computed using a variant of the \emph{Inside-Outside algorithm}~\cite{Lari1990}.
%
%\subsection{Probabilistic model}
%Let $\Omega$ be an gap-free RNA alignment sequence, $S$ its associated secondary structure, 
%then any sequence $s$ is assigned a probability proportional to its Boltzmann factor
%\begin{align*}
%  \mathcal{B}(s) &= e^\frac{-\PE{s}}{RT}, &&\text{with}&\PE{s}&:=\alpha\cdot\ES(s,S)+(1-\alpha)\cdot\EI(s,S,\Omega),
%\end{align*}
%where $R$ is the Boltzmann constant, $T$ the temperature in Kelvin, $\ES(s)$ and $\EI(s,S,\Omega)$ 
%are the free-energy and isostericity contributions respectively (further described below), and $\alpha\in[0,1]$ is an arbitrary parameter that sets the relative weight for both contributions.
%
%\subsubsection{Energy contribution}
\subsection{Definitions}

A targeted secondary structure $\Target$ is abstracted as a non-crossing arc-annotated sequence,  where 
$\Target_i$ stands for the base-pairing position of position $i$ in $\Target$ if any (and, reciprocally, $\Target_{\Target_i}=i$), or $-1$ otherwise. 
Let us denote by $\gc(s)$ the number of occurrences of \Gb and \Cb in an RNA sequence $s$.

\subsubsection{Simplified energy model}
The free-energy contribution in our pseudo-energy model corresponds to an additive stacking-pairs model, using individual values from the Turner 2004 model (retrieved from the NNDB~\cite{Turner2010}). Given a candidate sequence $s$ for a secondary structure $S$, the free-energy of $S$ on $s$ is given by
\begin{align*}
  \ES(s,S) = \sum_{\substack{(i,j)\to (i',j')\in S\\ \text{stacking pairs}}}\ES^{\beta}_{s_is_j\to s_{i'}s_{j'}} 
\end{align*}
where $\ES^{\beta}_{ab\to a'b'}$ is set to $0$ if $ab=\varnothing$ (no base-pair to stack onto), the tabulated free-energy of stacking pairs $(ab)/(a'b')$ in the Turner model if available, or $\beta\in[0,\infty]$ for non-Watson-Crick/Wobble pairs (i.e. not in $\{\Gb\Ub,\Ub\Gb,\Cb\Gb,\Gb\Cb, \Ab\Ub\text{ or }\Ub\Ab\}$). This latter parameter allows one to choose whether to simply penalize invalid base pairs ($\beta>0$), or forbid them altogether ($\beta = \infty$).

\subsubsection{\GCContent weighted pseudo-Boltzmann ensemble and distribution}

In order to counterbalance the documented tendency of sampling methods to generate \Gb\Cb-rich sequences~\cite{Levin:2012kx}, let us introduce a parameter $x\in\mathbb{R}^+$, whose value will influence the \GCContent of generated sequences. Let $\Target$ be a targeted secondary structure, then let us define the pseudo-Boltzmann factor $\B_{x}(s,S)$ of a sequence $s$ such that
\begin{equation}
\B_{x, \Target}(s) = e^{\frac{-\ES(s,\Target)}{RT}}\cdot x^{\gc(s)}
\label{def:genBoltz}
\end{equation}
where $R$ is the Boltzmann constant and $T$ the temperature in Kelvin.

Summing the pseudo-Boltzmann factor over all possible sequences of a given length $|\Target|$ gives the pseudo-partition function $\mathcal{Z}_{x,\Target}$, from which one defines the pseudo-Boltzmann probability $\Prob_{x,\Target}(s)$ of each sequence $s$, respectively such that 
\begin{align}\mathcal{Z}_{x,\Target} &= \sum_{\substack{|s|=|\Target|}}\B_{x,\Target}(s)& \text{and}&&
\Prob_{x,\Target}(s) &= \frac{\B_{x,\Target}(s)}{\mathcal{Z}_{x,\Target}}.\label{def:distribution}\end{align}

\subsection{Sampling the pseudo-Boltzmann ensemble}

Let us now describe a linear-time algorithm to sample sequences at random in the pseudo-Boltzmann distribution. This algorithm follows the general principles of the recursive approach to random generation~\cite{Wilf1977}, pioneered in the context of RNA by the \SFold algorithm~\cite{Ding2003}. The algorithm starts by precomputing the partition function restricted to each valid interval, and then performs a series of recursive stochastic backtracks, using precomputed values to decide on the probability of each alternative.

\subsubsection{Precomputing the pseudo-partition function}\label{sec:pf}
\begin{figure}
\resizebox{\textwidth}{!}{
\begin{tikzpicture}

  \newcommand{\BSep}{9pt}
  \newcommand{\HSep}{280pt}
  \newcommand{\VSep}{85pt}
  \tikzstyle{base}=[circle,draw,thick,inner sep=0,minimum width=15pt,fill=white]
  \tikzstyle{basesmall}=[circle,draw,thick,inner sep=0,minimum width=10pt,fill=white]
  \tikzstyle{basephantom}=[base,dashed]
  \tikzstyle{linez}=[draw,snake=coil, segment aspect=.2,%
line after snake=0pt, 
        segment length=10pt,thick]
  \tikzstyle{lined}=[linez,draw,snake=none,thick]
  \tikzstyle{line}=[linez,draw,snake=none,thick]
  \tikzstyle{bp}=[in=90,out=90,draw,line width=1.5pt,blue,looseness=1]
  \tikzstyle{block}=[trapezium,trapezium angle=33, fill=green!20, draw=green!40!black,line width=1.5pt, inner sep=0]
  \tikzstyle{lbl}=[inner sep=0]
  \tikzstyle{arr}=[line width=1.3pt,decorate,decoration={snake,amplitude=1mm, segment length=20pt,pre length=1mm, post length=2mm},->]

  \node[base] (a-1) at (0,0) {$s_i$};
  \node[base] (a-2) at (3,0) {$s_j$};
  \node[left=\BSep of a-1, basephantom] (a-0) {$a$};
  \node[right=\BSep of a-2, basephantom] (a-3) {$b$};
  \node[right=0 of a-3] (x) {};


  \path[linez] (a-1) --  (a-2);
  \path[lined] (a-0) -- (a-1);
  \path[lined] (a-2) -- (a-3);
  \node[lbl,below=3pt of a-1] (c1) {$i$};
  \node[lbl,below=3pt of a-2] (c2) {$j$};



  \begin{pgfonlayer}{background}
  \node[rectangle,inner sep=2pt,draw,fit=(a-1.west)(a-2.east) (c1) (c2)] (r1) {};
  \path[block]   (r1.south west) to (r1.south east) to (r1.north east) to[out=110,in=70] (r1.north west) to (r1.south west) ;
  \end{pgfonlayer}{background}


\begin{scope}[xshift=\HSep,yshift=\VSep]
  \node[base] (a-1) at (0,0) {\color{red}$a'$};
  \node[basesmall,right=\BSep of a-1] (a-1b)  {};
  \node[basesmall] (a-2) at (5,0) {};
  \node[left=\BSep of a-1, basephantom] (a-0) {$a$}; 
  \node[right=\BSep of a-2, basephantom] (a-3) {$b$};
  \path[linez] (a-1b) --  (a-2);
  \path[line] (a-1) -- (a-1b);
  \path[lined] (a-0) -- (a-1);
  \path[lined] (a-2) -- (a-3);
  \node[below=3pt of a-1b, inner sep=0] (c1) {$i+1$};
  \node[below=3pt of a-2, inner sep=0] (c2) {$j$};

  \node[right=3em of a-3,anchor=west] {\relsize{+2}[Position $i$ unpaired]};
  \node[xshift=-2pt] at (a-0.west) (y1) {};

\begin{pgfonlayer}{background}
  \node[rectangle,inner sep=2pt,draw,fit=(a-1b.west)(a-2.east) (c1) (c2)] (r1) {};
  \path[block]   (r1.south west) to (r1.south east) to (r1.north east) to[out=110,in=70,looseness=.6] (r1.north west) to (r1.south west) ;

  \end{pgfonlayer}{background}

\end{scope}

\begin{scope}[xshift=\HSep,yshift=-\VSep]
  \node[base] (a-1) at (0,0) {\color{red}$a'$};
  \node[base] (a-p) at (3,0) {\color{red}$b'$};

  \node[basesmall,right=\BSep of a-1] (a-1b) {};
  \node[basesmall,left=\BSep of a-p] (a-pb)  {};
  \node[basesmall,right=\BSep of a-p] (a-pa)  {};

  \node[basesmall] (a-2) at (5,0) {};
  \node[left=\BSep of a-1, basephantom] (a-0) {$a$};
  \node[right=\BSep of a-2, basephantom] (a-3) {$b$};

  \node[right=3em of a-3,anchor=west] {\relsize{+2}[Position $i$ paired anywhere else]};

  \path[linez] (a-1b) -- (a-pb);
  \path[linez] (a-pa) --(a-2);

  \path[line] (a-1) -- (a-1b);
  \path[line] (a-pb) -- (a-p);
  \path[line] (a-pa) -- (a-p);

  \path[lined] (a-0) -- (a-1);
  \path[lined] (a-2) -- (a-3);
  \node[lbl,below=3pt of a-1b] (c1) {$i+1$};
  \node[lbl,below=3pt of a-2] (c5){$j$};
  \node[lbl,below=3pt of a-p] (c3) {$k$};
  \node[lbl,below=3pt of a-pb] (c2){$k-1$};
  \node[lbl,below=3pt of a-pa] (c4){$k+1$};

  \draw[bp]  (a-1) to (a-p);
  \node[xshift=-2pt] at (a-0.west) (y2) {};

\begin{pgfonlayer}{background}

  \node[rectangle,inner sep=2pt,draw,fit=(a-1b.west)(a-pb.east) (c1) (c2)] (r1) {};
  \path[block]   (r1.south west) to (r1.south east) to (r1.north east) to[out=110,in=70,looseness=0.8] (r1.north west) to (r1.south west) ;

  \node[rectangle,inner sep=2pt,draw,fit=(a-pa.west)(a-2.east) (c4) (c5)] (r2) {};
  \path[block]   (r2.south west) to (r2.south east) to (r2.north east) to[out=110,in=70,looseness=0.8] (r2.north west) to (r2.south west) ;

  \end{pgfonlayer}{background}

\end{scope}

 
\begin{scope}[xshift=\HSep,yshift=00pt]
  \node[base] (a-1) at (0,0) {\color{red}$a'$};
  \node[base] (a-2) at (5,0) {\color{red}$b'$};

  \node[basesmall,right=\BSep of a-1] (a-1b) {};
  \node[basesmall,left=\BSep of a-2] (a-2b)  {};

  \node[left=\BSep of a-1, basephantom] (a-0) {$a$};
  \node[right=\BSep of a-2, basephantom] (a-3) {$b$};

  \path[linez] (a-1b) --  (a-2b);

  \path[line] (a-1) -- (a-1b);
  \path[line] (a-2) -- (a-2b);
  \path[lined] (a-0) -- (a-1);
  \path[lined] (a-2) -- (a-3);
  \node[lbl,below=3pt of a-1b] (c1) {$i+1$};
  \node[lbl,below=3pt of a-2b] (c2){$j-1$};

  \draw[bp,looseness=.7]  (a-1) to (a-2);
  \draw[bp,looseness=.7]  (a-0) to (a-3);
  \node[right=3em of a-3,anchor=west] {\relsize{+2}[Paired ends + Stacking pairs]};
  \node[xshift=-2pt] at (a-0.west) (y3) {};

\begin{pgfonlayer}{background}
  \node[rectangle,inner sep=2pt,draw,fit=(a-1b.west)(a-2b.east) (c1) (c2)] (r1) {};
  \path[block]   (r1.south west) to (r1.south east) to (r1.north east) to[out=110,in=70,looseness=.6] (r1.north west) to (r1.south west) ;
  \end{pgfonlayer}{background}

\end{scope}

  \tikzstyle{prob}=[fill=white,line width=1pt, opacity=.9]
  \path (x) edge[arr] %node[midway,prob]{$\frac{\mathcal{Z}\substack{(i+1,j)\\ [a',b]}}{\mathcal{Z}\substack{(i,j)\\ [a,b]}}$} 
(y1);
  \path (x) edge[arr] %node[midway,prob]{ $\frac{\mathcal{Z}\substack{(i+1,j)\\ [a',b]}}{\mathcal{Z}\substack{(i,j)\\ [a,b]}}$} 
(y2);
  \path (x) edge[arr] %node[midway,prob]{$\frac{\mathcal{Z}\substack{(i+1,j)\\ [a',b]}}{\mathcal{Z}\substack{(i,j)\\ [a,b]}}$} 
(y3);
  
\end{tikzpicture}}
\caption{Three main cases in the stochastic backtrack procedure over an interval $[i,j]$: Either position $i$ is left unpaired (top), a base-pair $(i,j)$ is formed, stacking onto $(i+1,j-1)$ (middle), or $i$ is paired somewhere else, defining two regions on which subsequent recursive calls are needed (bottom). Positions indicated in red are assigned during this stage of the backtrack.\label{fig:stochastic}}
\end{figure}


Firstly, a dynamic programming algorithm computes $\mathcal{Z}\substack{(i,j)\\ [a,b]}$ the pseudo-partition function restricted to an interval $[i,j]$, assuming its (previously chosen) flanking nucleotides are $a$ and $b$ respectively. 
Since the only sequence over an empty interval is the empty sequence, having energy $0$, one has
\begin{equation}
	\forall i \in [0,n-1]:\, \Z{i+1,i}{a,b}=1.
	\label{eq:Z_in}
\end{equation}


The recursion consists in four terms, depending on the base-pairing status and context of $(i,j)$:
\begin{equation}
	\Z{i,j}{a,b}:=\left\{
  \begin{array}{ll}
  		\displaystyle
      \sum_{a'\in \B}  
      x^{\gc(a')}
      \cdot\Z{i+1,j}{a',b} &\begin{array}{@{}l@{}}\text{If }\Target_{i}=-1\\ \text{\relsize{-1}Position $i$ unpaired,}\end{array}\\
      \displaystyle
      \sum_{a',b'\in \B^2}
			 x^{\gc(a'.b')}
			 \cdot e^{\frac{-\ES^{\beta}_{ab \to a'b'}}{RT}}
			 \cdot \Z{i+1,j-1}{a',b'}&
\begin{array}{@{}l@{}}\text{Elif }\Target_i=j \land \Target_{i-1}=j+1\\ \text{\relsize{-1}$(i,j)$ stacking pair onto $(i$-$1,j$+$1)$,}\end{array}
			 \\
			 \displaystyle
      \sum_{a',b'\in \B^2}
      x^{\gc(a'.b')}
			\cdot e^{\frac{-\ES^{\beta}_{\varnothing\to a'b'}}{RT}}
      \cdot\Z{i+1,k-1}{a',b'}
      \cdot\Z{k+1,j}{b',b} & 
\begin{array}{@{}l@{}}\text{Elif }\Target_i=k \land i < k \leq j\\
\text{\relsize{-1}Junction/bulge $\Rightarrow$ split, no stack,}
\end{array}
\\
      0 &\text{Otherwise.}
	\end{array}\right.
\label{eq:Z_rec}
\end{equation}

Despite its apparent quadratic complexity, the recurrence described in Subsection~\ref{sec:pf} can be computed in linear $\Theta(|\Target|)$ time and space, owing to the fact that, on any recursive call, the position $j$ is entirely determined by $i$. 

\begin{theorem}Starting from the complete region $[0,n-1]$, Equation~\eqref{eq:Z_rec} can be computed in $\Theta(|\Target|)$ time and space complexity.
\end{theorem}
\begin{proof}

Consider the {\bf covering base-pair} $\CNBP{i}$ of a position $i$, defined as the base-pair $(i',j')$ such that $i'<i<j'$ and which minimizes $i-i'$. For the sake of simplicity, and without loss of generality, let us assume that there exists a virtual base-pair $(-1,n)$ in $\Target$, so that any position in $[0,n-1]$ has a covering base-pair.
Then, it is easily shown by induction that, starting from the complete interval $[0,n-1]$, the recursive calls only evaluate $\mathcal{Z}$ over intervals $[i,j]$ such that $j=j'-1$, $\CNBP{i}=[i',j']$. 
\CFramed{This property clearly holds initially for the complete interval $[0,n-1]$, owing to the virtual base-pair $(-1,n)$. Furthermore, if it holds for a given $(i,j)$, $\CNBP{i}=[i',j+1]$, then:
\begin{itemize} 
  \item If $i$ is unpaired (Case 1), then $\CNBP{i+1}=\CNBP{i}=[i',j+1]$, and the property is satisfied by the subsequent call over $[i+1,j]$. 
  \item If $i$ is paired with $j$ and stacking onto $(i-1,j+1)$ (Case 2), then one has $\CNBP{i+1}=[i,j]$ and the  subsequent call over $[i+1,j-1]$ satisfies the property.
  \item If $i$ is paired with some $k$ (Case 3), then $\CNBP{i+1}=[i,k]$ and $\CNBP{k+1}=[i',j+1]$. Therefore the subsequent calls, respectively over $[i+1,k-1]$ and $[k+1,j]$, both satisfy the property.
\end{itemize}
The case of an empty sequence does not cause a recursive call and must not be verified.
}
Since, in any interval $[i,j]$ satisfying the property, the ending position $j$ is completely determined by $i$, then there exists at most $\Theta(|\Target|)$ intervals that need to be evaluated, and we conclude on the overall complexity with the remark that $a$, $b$, $a'$ and $b'$ vary over bounded ranges and only contribute to a constant factor in the complexity.
\end{proof}


\subsubsection{Stochastic backtrack}
Once the pseudo-partition functions have been memorized, a stochastic backtrack starts from the complete interval $[0,|\Target|-1]$ and, at each step, chooses a suitable assignment for one or several positions, using probabilities derived from the precomputation. One or several recursive calls over regions that are left unassigned are then performed, as described by Algorithm~\ref{alg:back}, and illustrated by Figure~\ref{fig:stochastic}.
\begin{algorithm}[t]
\DontPrintSemicolon
	\SetAlgoLined
\SetKwFunction{Backtrack}{Backtrack}
\SetKwFunction{Random}{Random}
	rand $\leftarrow$ \Random$\left(\Z{i,j}{a,b}\right)$\tcp*[r]{Draw random number in $[0,\Z{i,j}{a,b}[$}
 \lIf(\tcp*[f]{Empty sequence}){$j<i$}{\Return{$\varepsilon$}}
	\Else{
  k $\leftarrow S_i$\;
	\If(\tcp*[f]{Unpaired}){$(k = -1)$}{
		\For{$a'\in\B$}{
			rand $\leftarrow$ rand $- x^{\gc(a')}\cdot \Z{i+1,j}{a',b}$\;
			\lIf{$\text{rand}<0$}\Return{$a'.\Backtrack\left(i+1,j,a',b,\Target\right)$}\;
		}
	}
	\lElse{
		\If(\tcp*[f]{Stacking pairs}){$(k=j) \land (S_{i-1}=j+1)$}{
			\For{$(a',b')\in\B\times\B$}{
				rand $\leftarrow$ rand $-
			 	x^{\gc(a'.b')}
			 	\cdot e^{\frac{-\ES^{\beta}_{ab \to a'b'}}{RT}}
			 	\cdot \Z{i+1,j-1}{a',b'}	$\;
				\lIf{$\text{rand}<0$}{
					\Return{$a'.\Backtrack\left(i+1,j-1,a',b',\Target\right). b'$}\;		
				}
			}
		}
		\lElse{
			\If(\tcp*[f]{Paired, not stacked}){$(\Target_i=k) \land (i < k \leq j)$}{
				\For{$(a',b')\in\B\times\B$}{
					rand $\leftarrow$ rand $-	
       x^{\gc(a'.b')}
					\cdot e^{\frac{-\ES^{\beta}_{\varnothing\to a'b'}}{RT}}
  	    	\cdot\Z{i+1,k-1}{a',b'}
	    	  \cdot\Z{k+1,j}{b',b}$\;	
 					\If{$\text{rand}<0$}{
						\Return{$a'
						.\Backtrack\left(i+1,k-1,a',b',\Target\right)
						.b'
						.\Backtrack\left(k+1,j,b',b,\Target\right)
						$}\;	
					}
				}
		 }
		}
	}
 }
\caption{\protect\Backtrack$\left(i,j,a,b,\Target\right)$\label{alg:back}}
\end{algorithm}

Since, on each recursive call, the algorithm assigns at least one nucleotide to a -- previously unassigned -- position, and that the number of executions of the loops are bounded by a constant, then the complexity of the algorithm grows linearly on the targeted length. 


\subsubsection{Rejecting unsuitable candidates}

Note that, although the parameter $x$ is introduced to match the expected \GCContent with the targeted one, nothing prevents Algorithm~\ref{alg:back} to generate sequences of arbitrary \GCContent. Therefore, we use a rejection-based approach~\cite{Bodini2010} previously used by the authors in a similar context~\cite{Waldispuhl2011}.

It was established~\cite{Waldispuhl2011} that, for each value of $x$, there exists constants $\mu_x$ and $\sigma_x$ such that the \GCContent distribution observed in sequences sampled with respect to the above distribution converges asymptotically towards a normal law having expectation in $\mu\cdot n\cdot(1+o(1))$ and standard deviation in $\sigma\cdot\sqrt{n}\cdot(1+o(1))$.
Furthermore, it can easily be shown that the expected \GCContent is a continuous and strictly increasing monotonic  function of $x$, whose limits are $0$ when $x=0$ and $n$ when $x\to +\infty$. 

Consequently, for any targeted \GCContent $gc\in[0\%,100\%]$, there exists a unique value $x_{gc}$ such that generated sequences feature, on the average, the right \GCContent. Such a value can be determined using any numerical recipe, and a simple bisection procedure, further described in earlier work~\cite{Waldispuhl2011}, is used in our implementation. Furthermore, the concentration of the distribution, asserted by its limited standard deviation, implies a growth in $\Theta(\sqrt{n})$ (resp. $\Theta(1)$, i.e. constant) for the expected number of attempts before generating a sequence having the exact targeted \GCContent  (resp. within $[gc-\kappa,gc+\kappa]$, where $\kappa\in\Theta(1/\sqrt n)$).



\subsection{Postprocessing unpaired regions: A local/global (glocal) hybrid approach}
\label{subsec:glocal_method}
Due to our simplified our energy model, unpaired regions are not subject to design constraints other than the \GCContent, leading to modest probabilities for refolded design candidates to match the targeted structure. To improve these performances and test the complementarity  of our global sampling approach with previous contributions based on local search, we used the \RNAinverse software to redesign unpaired regions. We specified a constraint mask to prevent stacking base-pairs from being modified and, whenever necessary, reetablished their content content a posteriori as \RNAinverse has been witnessed to take some liberties with constraints masks. As shown in Table~\ref{table:impact_on_gc} (Supplementary material), this postprocessing does not drastically alter the \GCContent, so the glocal approach reasonably addresses the contrained \GCContent design problem.


