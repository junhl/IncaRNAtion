%!TEX root = main_ISMB.tex
\section{Conclusion}

\label{sec:conclusion}

In this article, we described a novel algorithm, \ourprog, for the RNA design problem, i.e. the design of an RNA sequence adopting a predefined secondary structure as its minimal free-energy fold.
Implementing a global sampling approach, it optimizes affinity towards the target secondary structure, while granting the user full control over the \GCContent of the resulting sequences.
This extended control does not necessarily induce additional computational demands, and we showed the linear complexity of both the preprocessing stage and the generation of candidate sequences for the design, allowing for the design of larger and more complex secondary structures in a matter of minutes on a single processor (e.g. $\sim$28 mins for 100 candidate sequences for a $\sim$1500nts 16s rRNA). We evaluated the method on a benchmark composed of target secondary structures extracted from the \RNASTRAND database. We observed good overall success rate, with the notable exception of very low targeted \GCContent ($10\%$), and a good to excellent entropy within designed candidates.
Finally, we implemented an hybrid approach, using the \RNAinverse software as a post-processing step for unpaired regions. This approach greatly increased the success rate of the method, allowing for the design of highly diverse candidates for almost all of the structures in our benchmark, while largely preserving the targeted \GCContent.

In the future, we would like to complement this study by further investigating the potential of hybrid local/global -- or {\em glocal} -- approaches.
A global sampling approach would capture the positive aspects of design, optimizing affinity towards a given structure while allowing the specification of expressive systems of constraints.
Designed sequences would serve as a seed for a restricted local approach which, by breaking unwanted symmetries, would perform the negative part of the design, 
while ideally maintaining obedience to the constraints. Another perspective of this work is the incorporation of the full Turner energy model, which should in principle yield better designs for unpaired regions.